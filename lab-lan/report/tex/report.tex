\documentclass[a4paper,12pt]{article}

% polyglossia should go first!
\usepackage{polyglossia} % multi-language support
\setmainlanguage{russian}
\setotherlanguage{english}
\PolyglossiaSetup{english}{indentfirst=true}
\PolyglossiaSetup{russian}{indentfirst=true}

\defaultfontfeatures{Mapping=tex-text} % for converting "--" and "---"
\setmainfont{CMU Serif}
\setsansfont{CMU Sans Serif}
\setmonofont{CMU Typewriter Text}

\usepackage{minted}
\newcommand{\inputmintedbr}[2]{\inputminted[breaklines=true]{#1}{#2}}

% Опционно, требует  apt-get install scalable-cyrfonts.*
% и удаления одной строчки в cyrtimes.sty
% Сточку не удалять!
% \usepackage{cyrtimes}

% Картнки и tikz
\usepackage{graphicx}
\usepackage{tikz}
\usetikzlibrary{snakes,arrows,shapes}


% Некоторая русификация.
%\usepackage{misccorr}
\usepackage{indentfirst}
\renewcommand{\labelitemi}{\normalfont\bfseries{--}}

% Увы, поля придётся уменьшить из-за листингов.
\topmargin -1cm
\oddsidemargin -0.5cm
\evensidemargin -0.5cm
\textwidth 17cm
\textheight 24cm

\sloppy

% Оглавление в PDF
\usepackage[
bookmarks=true,
colorlinks=true, linkcolor=black, anchorcolor=black, citecolor=black, menucolor=black,filecolor=black, urlcolor=black,
unicode=true
]{hyperref}

% Для исходного кода в тексте
\newcommand{\Code}[1]{\textbf{#1}}

%\usepackage{verbatim}
%\usepackage{fancyvrb}
%\fvset{frame=leftline, fontsize=\small, framerule=0.4mm, rulecolor=\color{darkgray}, commandchars=\\\{\}}
%\renewcommand{\theFancyVerbLine}{\small\arabic{FancyVerbLine}}

%tables
\usepackage{tabu}
\usepackage{multirow}

\title{Отчёт по лабораторной работе \\ <<Локальные сети>>}
\author{Гребенюк Александр Андреевич}

\begin{document}

\maketitle

\tableofcontents

% Текст отчёта должен быть читаемым!!! Написанное здесь является рыбой.

\clearpage
\section{Получение адреса по DHCP}

Получение ``случайного'' адреса \textbf{ws21} (дамп на \textbf{r2}):
\inputmintedbr{text}{../../results/ws21-dhcplease.log}

Получение ``фиксированного'' адреса \textbf{ws11} (дамп на \textbf{r1}):
\inputmintedbr{text}{../../results/ws11-dhcplease.log}


\clearpage
\section{Использование VPN}

Маршрутизатор \textbf{r1}:
\inputmintedbr{text}{../../results/r1.route}

\inputmintedbr{text}{../../results/r1-ipv4-a.log}

\inputmintedbr{text}{../../results/r1-rip.tun0.log}

Проверка работы VPN

\inputmintedbr{text}{../../results/ws21-s11.trace}


\clearpage
\section{Правила фильтации пакетов и трансляции пдресов}

Где что дампим. 

%\begin{Verbatim}
сценарий фильтрации
%\end{Verbatim}

%\begin{Verbatim}
iptables -L -nv
%\end{Verbatim}

%\begin{Verbatim}
iptables -L -nv -t nat
%\end{Verbatim}


\clearpage
\section{Проверка трансляции SNAT}

Где что дампим.

%\begin{Verbatim}
дамп SNAT в LAN (как вариант -i any tcp)
%\end{Verbatim}

%\begin{Verbatim}
дамп SNAT (снаружи)
%\end{Verbatim}


\clearpage
\section{Проверка правил фильтрации}

Используем telnet.


\clearpage
\section{Проверка доступа к внутреннему серверу}

Используем telnet / веб-браузер на реальной машине. 
Должен быть виден DNAT и разрешённый доступ.

\end{document}
