\documentclass[a4paper,12pt]{article}

% polyglossia should go first!
\usepackage{polyglossia} % multi-language support
\setmainlanguage{russian}
\setotherlanguage{english}
\PolyglossiaSetup{english}{indentfirst=true}
\PolyglossiaSetup{russian}{indentfirst=true}

\defaultfontfeatures{Mapping=tex-text} % for converting "--" and "---"
\setmainfont{CMU Serif}
\setsansfont{CMU Sans Serif}
\setmonofont{CMU Typewriter Text}

% Опционно, требует  apt-get install scalable-cyrfonts.*
% и удаления одной строчки в cyrtimes.sty
% Сточку не удалять!
% \usepackage{cyrtimes}

% Картнки и tikz
\usepackage{graphicx}
\usepackage{tikz}
\usetikzlibrary{snakes,arrows,shapes}


% Некоторая русификация.
%\usepackage{misccorr}
\usepackage{indentfirst}
\renewcommand{\labelitemi}{\normalfont\bfseries{--}}

% Увы, поля придётся уменьшить из-за листингов.
\topmargin -1cm
\oddsidemargin -0.5cm
\evensidemargin -0.5cm
\textwidth 17cm
\textheight 24cm

\sloppy

% Оглавление в PDF
\usepackage[
bookmarks=true,
colorlinks=true, linkcolor=black, anchorcolor=black, citecolor=black, menucolor=black,filecolor=black, urlcolor=black,
unicode=true
]{hyperref}

% Для исходного кода в тексте
\newcommand{\Code}[1]{\textbf{#1}}

\usepackage{listings}
\definecolor{bluekeywords}{rgb}{0.13,0.13,1}
\definecolor{greencomments}{rgb}{0,0.5,0}
\definecolor{turqusnumbers}{rgb}{0.17,0.57,0.69}
\definecolor{redstrings}{rgb}{0.5,0,0}
\definecolor{bluekeywords}{rgb}{0,0,1}
\definecolor{greencomments}{rgb}{0,0.5,0}
\definecolor{redstrings}{rgb}{0.64,0.08,0.08}
\definecolor{xmlcomments}{rgb}{0.5,0.5,0.5}
\definecolor{types}{rgb}{0.17,0.57,0.68}

\lstloadlanguages{bash}

\lstset{
  frame=none,
  xleftmargin=2pt,
  stepnumber=1,
  %numbers=left,
  %numbersep=5pt,
  %numberstyle=\ttfamily\tiny\color[gray]{0.3},
  belowcaptionskip=\bigskipamount,
  captionpos=b,
  escapeinside={*'}{'*},
  tabsize=2,
  emphstyle={\bf},
  commentstyle=\it,
  stringstyle=\mdseries\rmfamily,
  showspaces=false,
  keywordstyle=\bfseries\rmfamily,
  columns=flexible,
  basicstyle=\small\sffamily,
  showstringspaces=false,
  morecomment=[l]\%,
  breaklines=true,
  showlines=true,
  %language=bash
}
%\usepackage{verbatim}
%\usepackage{fancyvrb}
%\fvset{frame=leftline, fontsize=\small, framerule=0.4mm, rulecolor=\color{darkgray}, commandchars=\\\{\}}
%\renewcommand{\theFancyVerbLine}{\small\arabic{FancyVerbLine}}

%tables
\usepackage{tabu}
\usepackage{multirow}

\title{Отчёт по лабораторной работе \\ <<Динамическая IP-маршрутизация>>}
\author{Гребенюк Александр Андреевич}

\begin{document}

\maketitle

\tableofcontents

\section{Настройка сети}

\subsection{Топология сети}

Топология сети и используемые IP-адреса показаны на рисунке~\ref{fig:network}.

\begin{figure}
\centering
\includegraphics[width=0.8\textwidth]{includes/network_gv.pdf}
\caption{Топология сети}
\label{fig:network}
\end{figure}


Перечень узлов, на которых используется динамическая IP-маршрутизация: 
\begin{itemize}
  \item \textbf{r1-5},
  \item \textbf{wsp1}
\end{itemize}


\section{Назначение IP-адресов}
\begin{itemize}
\item Ниже приведён файл настройки протокола IP маршрутизатора \textbf{r1}.
\lstinputlisting{../../net/r1/etc/network/interfaces}
\item Ниже приведён файл настройки протокола IP маршрутизатора \textbf{r2}.
\lstinputlisting{../../net/r2/etc/network/interfaces}
\item Ниже приведён файл настройки протокола IP маршрутизатора \textbf{r3}.
\lstinputlisting{../../net/r3/etc/network/interfaces}
\item Ниже приведён файл настройки протокола IP маршрутизатора \textbf{r4}.
\lstinputlisting{../../net/r4/etc/network/interfaces}
\item Ниже приведён файл настройки протокола IP маршрутизатора \textbf{r5}.
\lstinputlisting{../../net/r5/etc/network/interfaces}

\item Ниже приведён файл настройки протокола IP рабочей станции \textbf{ws2}.
\lstinputlisting{../../net/ws2/etc/network/interfaces}
\item Ниже приведён файл настройки протокола IP рабочей станции \textbf{wsp1}.
\lstinputlisting{../../net/wsp1/etc/network/interfaces}
\end{itemize}



\subsection{Настройка протокола RIP}
\begin{itemize}
\item Ниже приведен файл \Code{/etc/quagga/ripd.conf} маршрутизатора \textbf{r1}.
\lstinputlisting{../../net/r1/etc/quagga/ripd.conf}
\item Ниже приведен файл \Code{/etc/quagga/ripd.conf} маршрутизатора \textbf{r2}.
\lstinputlisting{../../net/r2/etc/quagga/ripd.conf}
\item Ниже приведен файл \Code{/etc/quagga/ripd.conf} маршрутизатора \textbf{r3}.
\lstinputlisting{../../net/r3/etc/quagga/ripd.conf}
\item Ниже приведен файл \Code{/etc/quagga/ripd.conf} маршрутизатора \textbf{r4}.
\lstinputlisting{../../net/r4/etc/quagga/ripd.conf}
\item Ниже приведен файл \Code{/etc/quagga/ripd.conf} маршрутизатора \textbf{r5}.
\lstinputlisting{../../net/r5/etc/quagga/ripd.conf}
\end{itemize}


Ниже приведен файл \Code{/etc/quagga/ripd.conf} рабочий станции, связанной с несколькими маршрутизаторами \textbf{wsp1}.
\lstinputlisting{../../net/r5/etc/network/interfaces}


\section{Проверка настройки протокола RIP}

Вывод \textbf{traceroute} от узла такого-то до такого-то при нормальной работе сети.

\begin{Verbatim}
Сюда нужно поместить вывод traceroute.
\end{Verbatim}

Вывод \textbf{traceroute} от узла такого-то до внешнего IP (195.19.38.2 сгодится).

\begin{Verbatim}
Сюда нужно поместить вывод traceroute.
\end{Verbatim}

Вывод сообщения RIP.

\begin{Verbatim}
Перехваченное сообщение RIP от любого маршрутизатора
\end{Verbatim}

Вывод таблицы RIP.

\begin{Verbatim}
Таблица RIP
\end{Verbatim}

Вывод таблицы маршрутизации.

\begin{Verbatim}
Таблица маршрутизации
\end{Verbatim}

\section{Расщепленный горизонт и испорченные обратные обновления}

Поместить сюда вывод сообщения одного и того же маршрутизатор с включенным расщ. горизонтом, с включенными испорченными обновлениями, с отключённым расщ. гор.

Объяснить разницу.

Вернуть настройки в исходное состояние (включенный без испорченных).

\section{Имитация устранимой поломки в сети}

Какой маршрутизатор выключили?

Вывод таблицы RIP непосредственно перед истечением таймера устаревания (на маршрутизаторе-соседе отключенного).

\begin{Verbatim}
Таблица RIP
\end{Verbatim}

Перестроенная таблица на этом же маршрутизаторе

\begin{Verbatim}
Таблица RIP
\end{Verbatim}


Вывод \textbf{traceroute} от узла такого-то до такого-то после того, как служба RIP перестроила таблицы маршрутизации.

\begin{Verbatim}
Сюда нужно поместить вывод traceroute после "поломки".
\end{Verbatim}

\section{Имитация неустранимой поломки в сети}

Какой маршрутизатор выключили? (Теперь у нас нет связанной сети)

Далее поместить таблицы протокола RIP, где видна 16-ая метрика, и сообщения протокола RIP с 16-ой метрикой.

\end{document}
